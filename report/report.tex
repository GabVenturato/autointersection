\documentclass{memoir}

\title{Distributed Systems:\\Crossing intersection with autonomous vehicles}

\author{Antonio Toncetti\\Gabriele Venturato\\\\DMIF, University of Udine, Italy}

\date{%Version 0.1, 
	\today}

\begin{document}


%\begin{titlingpage}
\maketitle
\begin{abstract}
The aim of this project is to provide an implemented solution for the problem of crossing an intersection by autonomous vehicles.
Even with some simplifications, this can be considered a real-case scenario for this problem.

The solution proposed here try to be more general and modular as possible, in order to be possibly extended in a concrete situation.
\end{abstract}
%\end{titlingpage}

\end{document}

\chapter{Introduction}\label{ch:intro}

In this chapter you describe the main problem, and an idea of the solution.
It is not necessary to be very detailed or formal, but it is important to explain which are the main aims and issues from the point of view of Distributed Systems:
\begin{itemize}
\item A description of the application.
\item The overall structure of the implementation: how resources are deployed, which are the players, the r\^oles.
\item The distributed system features (and the transparencies) and algorithms you intend to implement.
\item Your plan for testing the system.
\item A schedule for how you plan to carry our your design and implementation
\end{itemize}

\chapter{Analysis}\label{ch:analysis}

In this chapter, we describe in detail functional and non-functional requirements of a solution for the problem.

\section{Functional requirements}
Which functions must be offered to users / other programs?  Which are the input data and the output data? Which is the expected effect? 

\section{Non functional requirements}
Everything about mode and transparencies: availability, mobility, security, fault tolerance, etc.

Are there execution time bounds? Minimum data rates?

If requested, specific platforms/languages/middlewares requirements for the implementation can be decided here. (E.g.: if the project is on a SOA, we may request that functions are offered via SOAP or RESTful services). 



\chapter{Project}

This chapter is devoted to the description of the general architectures, and specific algorithms.

\section{Logical architecture}
Describe the components of your systems: modules/objects/components/services.
For each component, describe the functionalities it implements, and by who is used.

\section{Protocols and algorithms}
Communication between components.  UML sequence diagrams go here.

Also, put here a detailed description of distributed algorithms used to solve specific problems of the project.

\section{Physical architecture and deployment}
Which nodes and platforms involved, and where each component is deployed.

\section{Development plan}
Since it is difficult to predict just how hard implementing a new system will be, you should formulate as a set of ``tiers,'' where the basic tier is something you’re sure you can complete, and the additional tiers add more features, at both the application and the system level.

\chapter{Implementation}

Details about the implementation: every choice about platforms, languages, software/hardware, middlewares, which has not been decided in the requirements.


Important choices about implementation should be described here; e.g., peculiar data structures.


\chapter{Validation}

Check if requirements from Chapter~\ref{ch:analysis} have been fulfilled.
Quantitative tests (simulations) and screenshots of the interfaces are put here.


\chapter{Conclusions}

What has been done with respect to what has been promised in Chapters~\ref{ch:intro} and \ref{ch:analysis}, and what is left out.

\appendix

\chapter{Appendix}

In the Appendix you can put code snippets, snapshots, installation instructions, etc.

\end{document}